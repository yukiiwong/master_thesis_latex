\chapter{仿真实验场景的设计与构建}

\section{城市交通仿真平台的可行性分析}

\subsection{Vissim介绍}

VISSIM(VISual SIMulation)是由德国PTV(Planung Transport Verkehr AG)公司开发的一款功能强大的交通模拟软件。它被广泛应用于交通规划、设计、管理等领域。VISSIM可以模拟各种复杂的交通场景,包括城市道路、高速公路、交叉口、公共交通,以及交通流、交通信号控制、公共交通路线规划、车辆路径规划等。
VISSIM的基本功能包括:

1.建模。VISSIM的建模功能非常强大。用户可以使用VISSIM的图形用户界面轻松创建交通场景,如道路、交叉口和公共交通路线。用户还可以调整交通流率、车辆类型、行人流量和其他参数,以建立一个交通网络。在VISSIM的智能交通生成器的帮助下,可以快速、准确地创建交通场景。交通生成器使用真实的交通数据,为一天中的不同时段、工作日和周末创建真实的交通流模式。交通生成器还允许用户定制交通情景,并调整交通量、速度和其他参数。

2.仿真模拟。VISSIM提供了一个高保真的交通仿真引擎,可以高度准确地模拟各种交通场景。用户可以运行各种交通管理策略的模拟,如交通信号控制、公共交通路线规划和车辆路径规划。仿真引擎还可以生成实时交通数据,如车辆速度、行驶时间和延迟时间。有了这些数据,用户可以评估不同交通管理策略的性能,比较不同的方案,并做出明智的决定。

3.分析功能。VISSIM提供了一个强大的分析功能,使用户能够分析和可视化模拟结果。用户可以生成大量的图表和表格来分析交通流量、拥堵情况、车辆行驶时间、车辆速度和其他指标。分析功能还允许用户比较不同交通管理策略的性能,评估不同参数对交通流的影响。

VISSIM是一个全面而强大的交通模拟软件,为用户提供准确而真实的模拟结果。该软件支持各种类型的交通场景,从小型交叉口到大型城市网络。建模功能允许用户创建一个虚拟的交通网络,模拟引擎产生的实时交通数据可用于评估不同的交通管理策略和交通计划。该软件的高级功能,如定制、多模式模拟、行人模拟、排放和油耗以及驾驶辅助系统,为用户提供了一套全面的工具来评估交通网络的性能和影响。总之,VISSIM是交通专业人员、研究人员和政策制定者设计、分析和优化交通系统的重要工具,以实现更安全、更高效和可持续的未来。


\subsection{SUMO介绍}

SUMO(Simulation of Urban Mobility)是一款开源的交通仿真软件,被广泛应用于城市交通规划、交通管理、交通研究等领域。SUMO是由德国科研机构DLR开发,其设计目标是实现高效、可扩展和高度自定义的交通仿真,同时提供良好的可视化和数据输出。SUMO具有开放性、高可靠性、高可扩展性和良好的可视化效果等特点,成为了交通仿真领域的重要工具之一。

SUMO的基本功能包括:

1.建模。SUMO提供了强大的建模功能,允许用户创建一个虚拟交通网络。该软件支持各种道路网络,包括高速公路、公路、交叉口和环岛等。用户可以定义网络中的道路布局、交通流和车辆类型。此外,该软件还提供了从各种文件格式(如OpenStreetMap)导入道路网络的工具。

2.仿真技术。SUMO使用微观交通模拟引擎,提供准确和真实的模拟结果。该软件可以模拟各种交通场景,如车辆路线、公共交通和行人运动。仿真引擎能够生成实时交通数据,如旅行时间、车辆速度和延迟时间。这些数据可以用来评估不同的交通管理策略和交通计划。

3.可视化。SUMO提供了一个用户友好的图形用户界面,允许用户对模拟结果进行可视化。该界面提供了一个交通网络的实时视图,并可以显示个别车辆、行人和公共交通的运动。此外,该软件还提供了生成各种图表和表格的工具,以总结模拟结果。

SUMO的高级功能有:

1.定制功能。SUMO是高度可定制的,允许用户定义影响模拟的不同参数,如车辆类型、司机行为和交通信号。该软件还提供了一个脚本接口,允许用户通过编写自定义脚本来扩展软件的功能。用户还可以创建自定义的车辆模型,并将其导入到模拟中。

2.多模式仿真。SUMO支持多模式模拟,允许用户模拟多种交通方式,如汽车、公交车、火车和自行车。这个功能可以评估不同交通方式的性能,并对不同的交通计划进行比较。

3.优化功能。SUMO提供优化功能,使用户能够改善交通网络的性能。该软件可以优化交通信号灯时间、车辆路线和公共交通时间表。优化功能允许用户找到最佳的交通管理策略和交通计划。

4.并行化。SUMO提供并行化功能,允许用户在多个处理器上运行模拟。这个功能加快了模拟时间,并能模拟更大的交通网络。

5.集成。SUMO可以与其他软件工具集成,如交通流模型、地理信息系统(GIS)和数据分析工具。集成功能使数据的交换和定制解决方案的开发成为可能。

总之,SUMO是一个功能强大、用途广泛的模拟软件,为用户提供了一套全面的工具来模拟和分析各种交通情况。该软件支持多模式交通的模拟,包括汽车、公交车、自行车和行人,它还提供了一些高级功能,如交通需求生成、交通信号控制和车辆路由。SUMO的灵活架构和开源代码库使其成为研究人员、交通专业人士和需要高度可定制模拟工具的决策者的热门选择。SUMO有能力生成真实的交通数据,并提供对交通系统性能的洞察力,在设计、优化和评估交通系统方面发挥了关键作用,以实现更安全、更高效和可持续的未来。

\subsection{Matsim介绍}

MATSim是一个开源的、基于代理的、多模式的交通模拟软件,为城市或地区的个人行为和整个交通系统的建模和模拟提供了一个平台。该软件由瑞士苏黎世联邦理工学院(ETH Zurich)的一个研究团队历经数年开发,目前已被世界各地的研究人员、交通规划人员和政策制定者广泛使用。

MATSim提供了一套全面的功能来模拟和仿真不同的交通场景。它将个人旅行者建模为代理人,根据他们的偏好和现有的交通选择做出决定。代理人可以选择不同的交通方式,如汽车、公共交通、步行或骑自行车,他们还可以选择最短、最快或最舒适的路线来到达目的地。该软件的先进模拟引擎可以生成大规模的模拟,可以在任何标准计算机系统上运行。

MATSim软件基于模块化架构,允许用户根据自己的具体需求定制软件。这些模块可以被扩展或被其他模块取代,以模拟新的运输系统或包括新的功能。这一特点使该软件能够适应不同交通场景的具体要求,并使其成为研究人员和交通专业人士的热门选择。

MATSim的主要特点之一是它能够模拟多模式的交通系统。它可以模拟广泛的交通选择,包括汽车、公共汽车、火车、自行车和步行,并且可以模拟不同交通方式之间的相互作用。这使用户能够评估不同交通系统的性能,并优化城市或区域内不同交通方式的使用。

MATSim还包括一个全面的可视化工具,允许用户实时查看模拟的结果。该软件的可视化模块使用户能够看到模拟的代理人在交通网络中移动,做出决定,并到达他们的目的地。可视化模块还可以显示模拟的交通流量、旅行时间和其他重要指标,可用于评估交通系统的性能。

MATSim的另一个主要特点是其开源代码库。该软件是在GNU通用公共许可证下发布的,它允许用户修改软件并将其分发给其他人。这一特点使研究人员和交通专业人士能够合作和分享他们的工作,使MATSim成为交通规划和管理领域的研究和开发项目的热门选择。

此外,MATSim软件已经由一个研究人员和交通专业人士团队进行了广泛的测试和验证。该软件已被用于对全球不同城市和地区的交通系统进行建模和模拟,包括苏黎世、柏林、悉尼、新加坡等。这有助于建立该软件的可信度和可靠性,并证明其对交通系统进行精确建模和模拟的能力。

总之,MATSim是一款功能强大、用途广泛的交通仿真软件,为用户提供了一套全面的工具来模拟和仿真不同的交通场景。该软件的模块化架构、多模式模拟能力、先进的可视化功能和开源代码库使其成为研究人员和交通专业人士的热门选择。MATSim能够准确地模拟交通系统并评估不同交通方案的性能,在设计、优化和评估交通系统以实现更安全、更高效和可持续的未来方面发挥了关键作用。


\subsection{仿真平台的选择}

在这三个交通模拟软件中,我会选择SUMO,原因有很多。

首先,SUMO是一个开源软件,这意味着它可以免费下载和使用。这对于预算紧张的人来说是一个很大的优势,因为使用SUMO没有任何许可费用。此外,SUMO的开源性质意味着它是高度可定制的,这使得用户可以根据他们的具体需求来修改软件。

其次,SUMO是一个高度通用的软件,可以模拟多模式的交通场景。它可以模拟汽车、公共汽车、自行车、行人,甚至火车。这意味着用户可以对各种交通场景进行建模和模拟,并评估不同交通方案的性能。

第三,SUMO是一个高度精确的模拟软件。该软件以先进的交通流和车辆动力学模型为基础,提供高度真实和准确的结果。这使得SUMO成为交通专业人士、研究人员和政策制定者的重要工具,他们需要评估交通系统的性能,并优化它们,以实现更安全、更高效和可持续的未来。

第四,SUMO有一个友好的界面,这使得初学者和高级用户都很容易使用。该软件的界面允许用户创建和修改交通网络,生成交通需求,并评估不同交通系统的性能。SUMO还提供了一套全面的可视化工具,使用户可以实时查看模拟结果。

第五,SUMO有一个庞大而活跃的用户社区,它提供了丰富的资源,包括教程、文档和用户论坛。这意味着,如果用户在使用该软件时遇到任何问题,可以很容易地找到支持和帮助。此外,用户社区还提供了一个合作和分享最佳实践和新发展的平台。

最后,SUMO是交通行业中备受推崇和广泛使用的交通仿真软件。它已被用于模拟和仿真世界上许多城市和地区的交通系统,包括欧洲、亚洲和北美。这意味着SUMO已经被交通专业人员广泛测试和验证,其结果也被政策制定者和决策者所信任。

总之,SUMO是一个强大的、多功能的交通仿真软件,为用户提供了一套全面的工具来模拟和仿真不同的交通场景。它的开源性、多模式模拟能力、准确性、用户友好的界面、活跃的用户社区和受人尊敬的声誉使它成为交通专业人士、研究人员和决策者的热门选择。凭借其准确模拟交通系统和评估不同交通方案性能的能力,SUMO在设计、优化和评估交通系统以实现更安全、更高效和可持续发展的未来方面发挥了关键作用。

\section{基于SUMO的城市交通仿真平台}

\subsection{平台设计目标}

本章旨在介绍使用SUMO仿真平台进行基于深度强化学习的出行模式和时间选择研究的设计目标。我们将首先概述本研究的研究背景和目的,然后介绍SUMO仿真平台的优点和特点,并说明我们将如何使用SUMO进行仿真实验,最后总结本章内容。

本研究旨在通过深度强化学习方法,探究出行者的出行模式和时间选择行为,并通过对不同出行者的学习,实现出行者之间的智能交互和协同,从而优化交通流量和提高出行效率。为此,我们选择SUMO仿真平台作为我们的仿真工具,以模拟不同的交通环境和交通管理策略,以及出行者之间的交互和协作。

SUMO仿真平台是一个开源的、高度可定制的交通流量仿真器,可以模拟不同的交通场景和交通管理策略。它提供了一个完整的仿真工具链,包括道路网络编辑器Netedit、仿真器SUMO、交互式仿真器SUMO-GUI和命令行接口TraCI等。SUMO支持多种车辆类型和行驶策略,如私家车、公共交通、自行车和行人等。同时,SUMO还支持多种路线选择算法和交通灯控制算法,用户可以选择最适合他们应用场景的算法。

在我们的研究中,我们将使用SUMO仿真平台来创建不同的交通环境和交通管理策略,并在其中加入深度强化学习算法来模拟出行者的行为。我们将使用SUMO-Tools和SUMO-Plugins等工具来分析和比较不同的交通管理策略和出行者行为模型,以评估其效果和优化方案。我们还将使用TraCI接口来控制和管理仿真场景,以模拟不同的交通环境和交通流量。

综上所述,本章介绍了使用SUMO仿真平台进行基于深度强化学习的出行模式和时间选择研究的设计目标。我们选择SUMO作为我们的仿真工具,以模拟不同的交通环境和交通管理策略,并使用深度强化学习算法来模拟出行者的行为。我们将使用SUMO-Tools和SUMO-Plugins等工具来分析和比较不同的交通管理策略和出行者行为模型,以评估其效果和优化方案。最终,我们希望通过本研究探究出行者之间的智能交互和协同,以优化交通流量和提高出行效率。通过使用SUMO仿真平台,我们可以创建复杂的交通场景,并添加自定义的功能和行为,以满足特定应用程序的需要。同时,SUMO还提供了丰富的数据输出和可视化功能,可以方便地对仿真结果和统计数据进行分析和可视化,为我们的研究提供了重要的支持和帮助。

在本章中,我们还将介绍我们的仿真平台设计的一些具体要素和技术细节,如如何设置仿真场景、如何定义出行者的行为模型、如何评估不同的交通管理策略等。我们将详细阐述如何使用SUMO-Tools和SUMO-Plugins等工具来分析和比较不同的交通管理策略和出行者行为模型,以及如何使用TraCI接口来控制和管理仿真场景。

综上所述,本章介绍了使用SUMO仿真平台进行基于深度强化学习的出行模式和时间选择研究的设计目标。我们选择SUMO作为我们的仿真工具,以模拟不同的交通环境和交通管理策略,并使用深度强化学习算法来模拟出行者的行为。通过本章的介绍,我们希望读者能够了解我们的仿真平台设计目标和技术细节,为后续章节的研究工作打下基础。
 

\subsection{功能模块简介}

Netedit是SUMO中一个重要的功能模块,它允许用户可视化创建、编辑和管理道路网络。Netedit提供了一个交互式的图形界面,用户可以通过鼠标和键盘输入,轻松地添加和编辑道路、路口、车道、交通灯等元素。Netedit支持多种地图格式,如OpenStreetMap、Google Maps和Bing Maps等。用户可以直接导入现有的道路网络或数据,也可以通过Netedit创建新的道路网络。

在Netedit中,用户可以通过简单的拖放操作和线条绘制工具添加道路和路口。用户可以定义每条道路的长度、车道数、最大速度和路权等信息。路口可以包括单向或双向转向,用户可以设置交通灯的定时和调度策略。用户还可以定义车辆行驶的规则和路权,以模拟不同的交通环境和交通管理策略。同时,Netedit还提供了一些有用的工具,如道路长度和宽度测量工具,帮助用户更加准确地绘制道路网络。

在创建和编辑道路网络之后,用户可以导出Netedit文件以供SUMO-GUI或其他仿真器使用。SUMO-GUI可以可视化地显示道路网络和交通流量,以便用户进行交通流量仿真和结果分析。Netedit还支持Python脚本,用户可以通过编写脚本来自动化地创建和编辑道路网络。这为大规模道路网络的创建和管理提供了便利。

综上所述,Netedit是SUMO中一个强大的道路网络编辑器,它提供了一个易于使用和直观的图形界面,允许用户轻松地创建、编辑和管理道路网络。通过Netedit,用户可以定义道路、路口、车道、交通灯等元素,以模拟不同的交通环境和交通管理策略。同时,Netedit还支持多种地图格式和Python脚本,为用户提供了更多的定制和扩展能力。

TraCI是SUMO中一个重要的功能模块,它允许用户创建和控制车辆和路口的运动和行为。TraCI可以通过Python脚本进行调用和控制,这使得用户可以模拟复杂的交通环境和交通管理策略。TraCI支持多种车辆类型和行驶策略,如私家车、公共交通、自行车和行人等。

TraCI提供了一系列API接口,用于访问和修改仿真器中的车辆和路口状态。用户可以查询车辆的位置、速度、加速度、目标路段等信息,并控制车辆的加速、刹车和转向。用户也可以控制路口的红绿灯和车辆进出等操作。TraCI还提供了一些有用的工具和功能,如路网查询、车辆生成和路由规划等,帮助用户更好地控制和管理交通流量。

TraCI可以与其他SUMO模块和工具集成,如SUMO-GUI、SUMO-Tools和SUMO-Plugins等。这使得用户可以创建复杂的交通管理策略和仿真场景,以模拟不同的交通环境和交通流量。同时,TraCI还支持可视化输出和数据记录,用户可以轻松地分析仿真结果和统计数据。

综上所述,TraCI是SUMO中一个功能强大的交通流量生成器,它允许用户创建和控制车辆和路口的运动和行为。通过TraCI,用户可以查询和修改车辆和路口状态,控制交通灯和车辆进出等操作,以模拟不同的交通管理策略和交通环境。TraCI还支持多种车辆类型和行驶策略,并可以与其他SUMO模块和工具集成,为用户提供更多的定制和扩展能力。

SUMO-Tools是SUMO中一个重要的功能模块,用于模拟和评估交通管理策略,如交通灯控制和路由优化。SUMO-Tools包括一个流量分析器、一个行驶速度分析器和一个决策支持工具等。用户可以使用SUMO-Tools分析和比较不同的交通管理策略,以评估其效果和优化方案。

SUMO-Plugins是SUMO的一个可扩展模块,允许用户添加自定义的功能和行为。它们可以是Python脚本、C++插件或Java插件等。用户可以使用SUMO-Plugins来创建和添加新的车辆类型、路线选择器、行驶策略和交通管理策略等。SUMO-Plugins还支持多种路线选择算法和交通灯控制算法,用户可以选择最适合他们应用场景的算法。

SUMO-Tools和SUMO-Plugins可以与其他SUMO模块和工具集成,如SUMO-GUI和TraCI等。用户可以使用SUMO-GUI可视化地显示仿真结果和统计数据,使用TraCI控制和管理仿真场景。同时,用户可以编写Python脚本和C++插件等扩展SUMO-Plugins,以满足特定应用程序的需要。这些功能模块和工具为用户提供了更多的定制和扩展能力,使得SUMO成为一个强大的交通模拟器。

综上所述,SUMO-Tools和SUMO-Plugins是SUMO中两个重要的功能模块,用于模拟和评估交通管理策略和扩展SUMO的功能和行为。它们可以与其他SUMO模块和工具集成,为用户提供更多的定制和扩展能力。通过使用SUMO-Tools和SUMO-Plugins,用户可以模拟和评估不同的交通管理策略,创建复杂的交通场景,并添加自定义的功能和行为,以满足特定应用程序的需要。


\section{实验场景的选择与搭建}

\subsection{路网的编辑与生成}

在本研究中,我们使用SUMO(Simulation of Urban MObility)作为交通仿真引擎。SUMO是一个开源的交通仿真软件,支持建模、仿真和分析各种交通场景,包括道路交通、公共交通、自行车交通和行人交通等。

为了进行基于深度强化学习的出行模式和时间选择研究,我们需要建立一个仿真环境来模拟交通流。在选择仿真区域方面,我们选择了中国苏州市的一个城市区域作为我们的仿真区域,该区域面积大约为20平方公里。这个区域是一个典型的城市区域,具有复杂的交通结构和各种出行方式。因此,这个区域是一个理想的研究对象,可以帮助我们更好地了解基于深度强化学习的出行模式和时间选择。

为了建立仿真环境,我们使用了OpenStreetMap(OSM)获取网络的几何和配置信息(2,423个节点和4,970条边),并将这些信息输入到SUMO中进行仿真。OSM是一个免费的、开源的地图服务,可以提供全球范围内的地图数据,包括道路、建筑、自然环境等。我们可以利用OSM提供的地图数据,轻松地建立仿真环境。

由于我们的仿真区域是一个多模式交通网络,公共交通(包括公交和地铁)也必须被配置。为了实现这一点,我们首先从地图服务应用程序中提取公共交通运营信息,然后进行地图匹配。提取的信息包括线路ID、停靠站或车站ID及其地理位置。这些信息可以通过地图匹配技术与实际地图中的位置进行匹配,从而实现公共交通在仿真环境中的配置。

最终,我们成功建立了一个具有复杂交通结构和多种出行方式的仿真环境,可以用于研究基于深度强化学习的出行模式和时间选择。通过SUMO的仿真功能,我们可以对不同出行模式和时间选择的影响进行量化分析,帮助我们更好地了解这些问题。

\subsection{出行模式的设计}

出行模式的设计是交通仿真中的一个重要方面。在本研究中,我们旨在研究深度强化学习在出行模式和出发时间选择方面的应用。因此,我们需要在SUMO交通仿真软件中建立一个具有多种出行方式的仿真环境。

在本研究中,我们考虑了私家车、公共交通(包括公交车和地铁)和自行车三种出行方式。私家车是最常见的出行方式之一,而公共交通和自行车则是城市交通中的重要组成部分。为了实现这种多模式交通网络的仿真,我们需要对这些出行方式进行适当的建模和参数化。

首先,我们需要确定私家车、公共交通和自行车在SUMO仿真中的特征和参数。对于私家车,我们需要考虑其速度、加速度和刹车距离等因素。对于公共交通,我们需要考虑车辆的数量、停靠站点、运营时间和乘客人数等因素。对于自行车,我们需要考虑其速度、加速度和行驶距离等因素。

其次,我们需要将这些特征和参数纳入到SUMO仿真中,以便代理可以在仿真环境中对它们进行操作。对于私家车,我们可以利用SUMO的Car-Following模型来建模其运动行为。对于公共交通,我们可以利用SUMO中的公交车和地铁模型来建模其运营行为。对于自行车,我们可以利用SUMO的Bicycle模型来建模其运动行为。通过将这些模型组合在一起,我们可以实现多种出行方式的仿真。

最后,我们需要在SUMO仿真环境中建立一个多模式交通网络。这个网络需要包括私家车、公共交通和自行车在内的所有出行方式,以便代理可以在仿真环境中对它们进行操作。我们需要将道路、交叉口和停靠站点等元素纳入仿真环境中,并通过SUMO的网络编辑工具对其进行编辑和配置。这样,我们就可以建立一个具有多种出行方式的仿真环境,用于研究基于深度强化学习的出行模式和出发时间选择。

在本研究中,我们成功地建立了一个具有复杂交通结构和多种出行方式的仿真环境,可以用于研究出行模式和出发时间选择。通过SUMO的仿真功能,我们可以对不同出行模式和出发时间选择的影响进行量化分析,帮助我们更好地了解这些问题。同时,我们还可以通过仿真来测试不同的出行政策和规划方案,以便为决策者提供科学的依据和参考。

为了建立融合公共交通、私家车和自行车的多模式交通网络,我们需要对网络进行适当的编辑和生成。在本研究中,我们使用OpenStreetMap(OSM)来获取网络的几何和配置信息,并将这些信息输入到SUMO中进行仿真。在OSM中,道路、路径、建筑和其他空间要素都被准确地绘制在地图上。SUMO可以将这些信息转换为交通仿真模型,并在模拟中使用它们。同时,我们还需要配置公共交通(包括公交车和地铁)和自行车的路线和停靠点。这些信息可以从地图服务应用程序中提取,并与OSM中的信息进行匹配。

在多模式交通网络中,不同的出行模式需要不同的道路和路径支持。例如,自行车需要有专门的自行车道,而公共交通需要有专门的公交车道和地铁隧道。在网络生成过程中,我们需要将这些要素考虑在内,并将其配置到对应的道路和路径上。这样,在模拟中,不同的出行模式可以根据自己的需求使用不同的道路和路径,从而更好地模拟现实世界中的交通流。

在多模式交通网络中,出行者的出行方式选择涉及到多个变量和因素。例如,私家车出行者可能会考虑到停车地点和停车费用,而公共交通出行者可能会考虑到车站或站点的位置和交通运行时间等。在模型设计中,我们需要考虑到这些因素,以便为出行者提供准确的出行方式选择。这些因素可以编码为模型的状态变量,并在代理决策时提供给代理。

总之,通过在SUMO中建立融合公共交通、私家车和自行车的多模式交通网络,我们可以更好地研究出行模式和出发时间选择问题。这种仿真方法可以提供高度可控的实验环境,并帮助我们更好地理解交通流的动态变化和出行者的行为决策。

\subsection{流量的生成}
为了进行交通仿真实验,我们需要生成一定的交通流量。本研究中,我们选择典型的早高峰时段(上午7点至9点)作为研究对象,考虑了五个连续工作日内的出行模式和时间选择。每个工作日被视为训练过程中的一个episode。虽然我们使用了合成的旅行需求,但这并不影响所提出方法的有效性。仿真时段为2小时,被分为4个30分钟的时间间隔。对于每个工作日,我们有一个典型的早高峰出行需求模式,对于每个时间间隔,出行需求被视为一个遵循高斯分布的随机变量(见表1)。因此,在仿真过程中,每个episode都与略有不同的出行需求相关联。这种随机性符合现实中观察到的某种模式下的交通流量日常波动(即周期性拥堵)。然而,在非周期性拥堵或意外事件的情况下,所提出的方法可能不再适用,因为这些事件对出行选择的影响没有被体验和学习。在这方面进行更进一步的研究是值得的。

我们使用OpenStreetMap(OSM)获取网络的几何和配置信息(2,423个节点和4,970条边),并将这些信息输入到SUMO中进行仿真。同时,由于我们的仿真区域是一个多模式交通网络,因此我们需要配置公共交通(包括公交和地铁)。为了实现这一点,我们首先从地图服务应用程序中提取公共交通运营信息,然后进行地图匹配。提取的信息包括线路ID、停靠站或车站ID及其地理位置。这些信息可以通过地图匹配技术与实际地图中的位置进行匹配,从而实现公共交通在仿真环境中的配置。

表1展示了我们对五个连续工作日内早高峰时段的交通需求配置。表格中列出了每个时间段的车辆数,以及使用的随机分布。我们将每个工作日的交通需求视为随机变量,通过高斯分布模拟早高峰期间的交通需求波动。这种方式可以使得每个episode的交通需求略有不同,从而更贴近实际的交通流量波动情况。