\chapter{深度强化学习}
\label{chp:initialization}

\section{强化学习}

强化学习是一种基于马尔可夫决策过程的算法。在规定的策略中,智能体根据现处的状态通过动作与环境进行交互,并产生新的状态,同时从环境得到奖励。重复循环此过程,直至智能体完成设定的目标。强化学习算法就是利用在不断探索环境的过程中利用产生的奖励数据优化其行为策略,以获得最大的回报。本节将首先介绍强化学习算法的相关术语。之后,根据智能体动作选取方式,将强化学习方法分为基于价值、基于策略,以及基于价值和策略三类分别综述。
\subsection{相关术语}
智能体:任何有独立思想并且可以与所处环境进行交互的实体。在交通场景下,智能体可以是行人,车辆,信号灯等。

状态:当前时刻智能体对周围环境的感知。所有时刻的感知集合构成了状态空间。

动作:智能体在当前状态下采取的行动。在当前状态下能采取的所有动作构成了动作空间。

策略:智能体根据当前时刻的状态应采取哪个动作的控制准则。在数学上的含义为,使用概率密度函数表示智能体在每个状态下采取各个动作的概率。

奖励:在智能体采取动作后,环境对智能体的反馈效果。奖励R可以为正反馈或负反馈。

回报:智能体从当前时刻至行动结束所能获得的累积奖励之和。

状态转移:当智能体采取动作后,由当前状态转移到下一个状态的过程。状态转移过程大多数具有随机性,该随机性来源于环境。
\subsection{基于价值的强化学习}

基于价值的强化学习使智能体通过行动与预期结果(奖励)联系起来,通过试验和错误进行学习。智能体的主要目标是通过学习在不同情况下采取的最佳行动,随着时间的推移使其累积奖励最大化。在基于价值的强化学习中,智能体学习预测在特定状态下采取特定行动的价值。一个行动的价值通常被定义为智能体在特定状态下采取该行动并遵循特定政策所能获得的预期累积奖励。

在强化学习中,任意t时刻的状态$s_t$下执行策略$\pi$中的动作$a_t$都存在一个对应的奖励$R_t$,由于强化学习所研究的问题具有马尔可夫性,因此系统的整体回报$U_t$与当前时刻的奖励$R_t$和未来时刻的奖励$R_{t+n}$有关,所以存在等式:
\begin{equation}
  \label{eq:2_1}
  U_t = R_t + \gamma R_{t+1} + \gamma^2 R_{t+2} + ... + \gamma^n R_{t+n}
\end{equation}
式中,$\gamma$是折减因子。

在$t$时刻的回报$U_t$中,未来的奖励是与未来的状态和动作相关,而两者都具有随机性,所以需要通过对$U_t$求解期望值$Q_π (s_t,a_t)$来消除随机性。
\begin{equation}
  \label{eq:2_2}
  Q_π (s_t,a_t) = E[U_t\mid S_t=s_t, A_t = a_t]
\end{equation}

因此,$Q_π (s_t,a_t)$的值可以用来描述状态动作对$(s_t,a_t )$的价值。其中,$Q(s,a)$就是该强化学习的动作价值函数。
通过寻找$t$时刻所有策略$\pi$的动作价值函数的最大值,可以得到最优的策略$\pi$的动作价值函数$Q^* (s_t,a_t )$。
\begin{equation}
  \label{eq:2_3}
  Q^* (s_t,a_t) = \max Q_π (s_t,a_t)
\end{equation}

对最优策略$\pi$中的动作集$A$取最大值,即可获取每一次的最优动作$a^*$。
\begin{equation}
  \label{eq:2_4}
  a^* = \argmax Q^* (s_t,a_t)
\end{equation}

在基于价值的强化学习模型中,其主要目的就是逼近最优的策略π的动作价值函数$Q^* (s_t,a_t )$。可以利用神经网络等方法近似动作价值函数进行求解。

由式\ref{eq:2_4}中动作价值函数$Q^* (s_t,a_t )$可以得到价值最高的动作空间$A^*$。在强化学习中,一般使用神经网络的方法近似函数$Q^* (s_t,a_t )$,网络的输入为状态$s$,网络的输出为不同动作的价值。则有:
\begin{equation}
  \label{eq:2_5}
  Q(s,a;\mathbf{w}) \rightarrow Q(s,a)
\end{equation}

	式中,$\mathbf{w}$是价值网络(Value Network)的参数。可以通过不同状态下的奖励$R$利用时序差分算法更新价值网络,使得网络的参数$\mathbf{w}$更加精确。
\begin{equation}
  \label{eq:2_6}
  Q(s,a;\mathbf{w}) \approx R_t + \gamma \cdot Q(s,a;\mathbf{w})
\end{equation}

最常见的基于价值的强化学习算法是Q-learning。Q-learning是一种估计最佳动作价值函数的无模型方法,它代表了智能体在特定状态下采取特定动作并遵循最佳策略所能获得的预期累积奖励。一个状态-行动对的Q值使用贝尔曼方程进行更新,该方程指出,一个状态-行动对的最佳Q值等于即时奖励加上折现的最大预期未来奖励。Q-learning是一个迭代过程,包括在智能体采取每个行动后更新Q值,并接受奖励形式的反馈。随着时间的推移,智能体学会了所有状态-行动对的最佳Q值,使其能够在每个状态下选择最佳行动,使其累积奖励最大化。

基于价值的强化学习方法已经在各种应用中取得了巨大的成功,包括游戏、机器人和自动驾驶汽车,使得智能体能够学习如何在复杂和不确定的环境中做出最佳决策。


\subsection{基于策略的强化学习}

基于策略的强化学习主要是为智能体在环境中采取行动寻找最佳策略,以使奖励最大化。策略是一种从状态到行动的映射,它告诉智能体在特定状态下应采取何种行动。基于策略的强化学习的目标是找到一个策略,使智能体的预期奖励在一段时间内最大化。在强化学习中,使用概率密度函数$\pi(a│s)$来控制智能体在不同状态下的动作选取,即策略函数。策略函数的输入为当前$t$时刻的状态$s_t$,输出为所有动作的概率值。依据策略函数得到的概率值对所有动作随机抽样后,确定在状态$s_t$下进行的动作$a_t$。当使用神经网络的方法近似策略函数时,则有:
\begin{equation}
\label{eq:2_7}
\left\{\begin{array}{l}
\pi(a \mid s ; \boldsymbol{\theta}) \rightarrow \pi(a \mid s) \\
\sum_{a \in A} \pi(a \mid s ; \boldsymbol{\theta})=1
\end{array}\right.
\end{equation}

式中,$\boldsymbol{\theta}$是策略网络的参数。

通过式\ref{eq:2_2},对$Q_\pi\left(s_t,a_t\right)$求取期望,通过积分消除概率密度函数$\pi\left(\bullet\middle| s\right)$中的动作A可以得到状态价值函数$V_\pi$:
\begin{equation}
\label{eq:2_8}
V_\pi\left(s_t\right)=E_A[Q_\pi (s_t,A)]
\end{equation}

状态价值函数$V_\pi\left(s_t\right)$只与当前策略$\pi$和状态$s_t$有关。因此,状态价值函数可以用来评价当前状态下不同策略的价值。
	如果是离散的动作空间,状态价值函数$V_\pi\left(s_t\right)$可以写作:
\begin{equation}	
\label{eq:2_9}
V_\pi\left(s_t\right)=\sum_{a}{\pi\left(a\middle| s_t\right)\cdot}Q_\pi\left(s_t,a\right)
\end{equation}

如果是连续的动作空间,则使用积分形式代替连加求和。由于连续动作空间的研究较复杂,并且大多数可以离散化,因此之后均为离散动作空间下的状态价值函数。通过式\ref{eq:2_7}中策略网络近似得到的策略函数$\pi(a\mid s_t;\boldsymbol{\theta})$,可以近似状态价值函数:
\begin{equation}	
\label{eq:2_10}
V(s ; \boldsymbol{\theta})=\sum_a \pi(a \mid s ; \boldsymbol{\theta}) \cdot Q_\pi\left(s_t, a\right)
\end{equation}

基于策略的方法通常使用随机梯度上升法来更新策略。策略$\pi$由一组参数表示,使用策略梯度定理等技术计算出相对于这些参数的预期奖励的梯度。然后使用梯度上升法更新参数,以改进策略。策略学习是通过学习式\ref{eq:2_9}中的参数$\boldsymbol{\theta}$,得到价值最高的策略。这个过程中需要通过不断地改进策略网络参数$\boldsymbol{\theta}$的使$V(s;\boldsymbol{\theta})$的值达到最大值。因此,可以将式\ref{eq:2_9}中的$V(s;\boldsymbol{\theta})$对状态空间S求期望,将目标函数转化为$J(\boldsymbol{\theta})$:
\begin{equation}	
\label{eq:2_11}
J(\boldsymbol{\theta})=E_S[V(S;\boldsymbol{\theta})]
\end{equation}

基于策略的强化学习的缺点之一是它的计算成本很高,因为策略通常由大量的参数表示。此外,策略有时会卡在局部最优处,这可能使它难以找到全局最优策略。

总的来说,基于策略的强化学习是一种在复杂和动态环境中寻找最优策略的强大方法,并且已经成功地应用于广泛的领域,包括机器人、游戏和自然语言处理。

\subsection{价值与策略相结合的强化学习方法}

在强化学习中,将策略网络与价值网络同时训练更新的方法称为策略价值结合学习方法。其目的为使智能体通过策略网络做出的动作价值越来越高的同时,使得价值网络对动作价值的评价越来越精准。在策略价值结合学习方法中,可以把策略网络当作行动者(actor),价值网络当作裁判(critic)。价值网络会对智能体通过策略网络做出的动作进行评价,帮助更新策略网络参数,

通过联立式\ref{eq:2_5}与式\ref{eq:2_10},可以得到通过神经网络方法近似后的价值函数$Q(s,a;\boldsymbol{w})$与策略函数$\pi(a\mid s;\boldsymbol{\theta})$。因此,状态价值函数可以写作:
\begin{equation}
\label{eq:2_12}
\left\{\begin{array}{c}
V(s ; \boldsymbol{\theta}, \boldsymbol{w})=\sum_a \pi(a \mid s ; \boldsymbol{\theta}) \cdot q(s, a ; \boldsymbol{w}) \\
\sum_{a \in A} \pi(a \mid s ; \boldsymbol{\theta})=1
\end{array}\right.
\end{equation}

此时,可以把策略网络当作行动者(actor),价值网络当作批评者(critic)。价值网络会对智能体通过策略网络做出的动作进行评价,帮助更新策略网络参数,使其目标函数$J(\boldsymbol{\theta})$的值更大。行动者和批评者根据奖励和估计的状态-行动值进行更新。批评者通过最小化估计值和真实值(奖励和下一个状态的估计值之和)之间的平均平方误差来更新其对状态行动值的估计。行动者以估计的状态行动值为指导,通过最大化预期收益(未来奖励的总和)来更新其政策。

与其他强化学习算法相比,通过学习策略和价值函数,价值与策略相结合的强化学习方法可以比基于策略的方法更快地收敛,比基于价值的方法更稳定。它还可以处理高维的状态和行动空间,并且可以在实时环境中在线学习。价值与策略相结合的强化学习方法结合了基于政策和基于价值的方法的优点,可以同时学习最优政策和最优价值函数。然而,该方法需要仔细调整学习率和其他超参数以确保稳定的学习,而且它可能存在收敛问题和价值函数估计的偏差。


\section{深度学习}

\subsection{神经网络}

神经网络是一种机器学习模型,其灵感来自于人脑的结构和功能。它是由多个相互连接的节点或神经元组织成层,并形成一个系统。每个神经元接收来自其他神经元的输入,处理输入数据后产生一个输出信号。然后一个层的输出被用作下一层的输入,直至最终层输出结果。神经网络在训练期间从数据中学习经验并调整神经元之间连接的权重。权重决定了神经元之间的连接强度,它们使用优化算法进行更新,以达到预测输出和实际输出之间的误差最小化。神经网络已被应用于广泛的场景中,包括图像和语音识别、自然语言处理以及时间序列预测等。目前在深度学习领域内使用的主流神经网络结构有前馈神经网络、递归神经网络和卷积神经网络。

最常用的架构是前馈神经网络,其输入数据是沿同一个方向流经各层。前馈神经网络主体是由一个输入层、多个隐藏层和一个输出层组成。各个层的职责各不相同:输入层接收输入数据,输出层产生神经网络的最终输出,隐藏层负责学习输入数据的特征。输入层的每个神经元代表输入数据的一个特征,而输出层的每个神经元代表神经网络预测的一个类别或一个值。隐蔽层中每个神经元的输出是通过对输入和权重的线性组合来计算的,并加入一个偏置项。然后,输出通过一个激活函数,将非线性引入网络。激活函数将神经元的输出转换到一个特定的范围,通常在0到1或-1到1之间。最常用的激活函数是ReLU、sigmoid和tanh。

隐藏层中每个神经元的输出可以按以下方式计算。

\begin{equation}
\label{eq:2_13}
\left\{\begin{array}{c}
z = w * x + b \\
a = f(x)
\end{array}\right.
\end{equation}

其中$z$是输入和偏置的加权和,$w$是权重向量,$x$是输入向量,$b$是偏置项,$f$是激活函数,$a$是神经元的输出。

然后,隐藏层中每个神经元的输出被用作下一层的输入。最后一层的输出是神经网络的最终输出。在训练过程中,神经网络通过调整神经元之间连接的权重和偏差,使预测输出和实际输出之间的差异最小。

反向传播是一种用于训练神经网络的算法,通过调整神经元之间连接的权重和偏置项来训练。它是一种基于梯度的优化算法,计算损失函数相对于权重和偏置的梯度,并按照负梯度的方向更新它们。其中,损失函数衡量的是预测输出和实际输出之间的误差。回归问题最常用的损失函数是平均平方误差,而分类问题则使用交叉熵损失。损失函数相对于权重和偏差的梯度可以用微积分的链式法则来计算。链式法则指出,一个复合函数的导数等于其组成部分的导数的乘积。在神经网络的背景下,链式法则被用来计算损失函数相对于每个神经元输出的导数,然后通过网络传播误差来调整权重和偏差。

损失函数对于神经元输出的梯度可以按以下方式计算。
\begin{equation}
\label{eq:2_14}
\frac{\partial L}{\partial a} = \frac{\partial L}{\partial z} * \frac{\partial z}{\partial a}
\end{equation}


其中$L$是损失函数,$a$是神经元的输出,$z$是输入和偏置项的加权和。

式\ref{eq:2_14}右边的第一项是损失函数相对于加权和的导数,可以用激活函数的导数来计算。第二项是加权和相对于神经元输出的导数,也就是权重向量。然后,损失函数相对于权重和偏置的梯度可以通过在神经网络中各神经层向后传播误差来计算。首先计算输出层的误差,然后使用链式法则通过隐藏层向后传播,最后使用计算出的梯度和学习率更新权重和偏置项。权重通常在训练前被随机初始化。学习率决定了权重和偏置更新的步长,通常使用试验和错误或网格搜索来选择。高的学习率可能会导致对最优权重的过度拟合,而低的学习率则会导致缓慢的收敛。

训练神经网络的挑战之一是过拟合,即模型学习到的数据过于适合训练数据而在新数据上表现不佳。当模型相对于可用于训练的数据量来说过于复杂时,就会出现过拟合。正则化技术可以用来防止过度拟合。最常用的正则化技术是L1和L2正则化。L1正则化给损失函数增加了一个惩罚项,与权重的绝对值成正比,而L2正则化则是增加了一个惩罚项,与权重的平方成正比。惩罚项的作用是鼓励权重变小,这有助于防止过度拟合。Dropout是另一种正则化技术,它在训练过程中随机丢弃一部分神经元,这种做法可以使得剩余的神经元学习更平稳的特征。Dropout可以被认为是一个具有共享权重的几个神经网络的集合。


\subsection{激活函数}

激活函数是神经网络的一个关键组成部分,如果没有激活函数,神经网络本质上只是线性回归模型,这将严重限制发挥其能力和灵活性。激活函数是应用于神经网络中每个神经元的输出的函数。激活函数的目的是在模型中引入非线性,这对于捕捉数据中的复杂模式来说是必要的。其计算过程主要是在神经元输入的加权和添加一个偏置项后,对结果进行非线性转换。之后,激活函数的输出值将被传递到网络的下一层作为输入数据。目前在深度学习中应用最广泛的激活函数有Sigmoid函数,ReLU函数和Softmax函数。

sigmoid函数是一条平滑的S形曲线,接受任何输入并输出0到1之间的值。sigmoid函数的公式如下:
\begin{equation}
\label{eq:2_15}
f(x) = 1 / (1 + e^-x)
\end{equation}


其中$x$是该函数的输入。

Sigmoid函数是可微的,这意味着它的导数可以在任何一点处计算出来,这使得它很适合用于反向传播,也就是用于训练神经网络的算法。此外,Sigmoid函数在0和1之间是有界的,这意味着它可以被解释为一个概率。然而,Sigmoid函数也有一些缺点。Sigmoid函数的主要问题之一是它存在梯度消失的问题。当Sigmoid函数的输入非常大或非常小时,函数的输出分别变得非常接近于0或1,函数的导数也会变得非常小,这可能导致梯度在反向传播期间消失。因为梯度在网络中向后传播时变得非常小,这样会使得网络难以学习更加深度的表征。


ReLU函数是神经网络中另一个常用的激活函数。它是一个片状线性函数,接受任何输入,如果输入是正的,就输出,否则就是0。ReLU函数的公式如下:

\begin{equation}
\label{eq:2_16}
f(x) = \max(0, x)
\end{equation}


其中$x$是该函数的输入。

ReLU函数的主要优点之一是它不存在梯度消失的问题。当ReLU函数的输入为正数时,该函数的导数为1,这意味着在反向传播过程中,梯度仍然很大。这使得网络更容易学习深度表征,因为梯度在通过网络向后传播时不会消失。ReLU函数的另一个优点是它的计算效率高。由于该函数只是一个阈值操作,它可以用简单的逻辑运算来实现。然而,ReLU函数也有一些缺点。ReLU函数的一个主要问题是,它受到垂死的ReLU问题的影响。当ReLU函数的输入为负数时,该函数的输出为0,这意味着神经元实际上变得不活跃。这可能会导致整个神经元在训练过程中死亡,这可能会对网络的性能产生负面影响。


Softmax函数是一种特殊的激活函数,常用于用于分类任务的神经网络的输出层。Softmax函数接收一个输入矢量,并输出一个和为1的数值矢量,可解释为概率。

softmax函数由以下公式给出。

\begin{equation}
\label{eq:2_16}
f(x_i) = e^(x_i) / \sum _j(e^(x_j))
\end{equation}



其中$x_i$是输入矢量的第$i$个元素,和是在矢量的所有元素上取的。

softmax函数有几个有用的特性。softmax函数的主要好处之一是,它确保网络的输出可以被解释为概率。这使得它非常适合用于分类任务,其目标是将输入分配到几个可能的类别中的一个。

此外,softmax函数是可微分的,这意味着它可以用于反向传播来训练网络。

其他激活函数。

除了上面讨论的激活函数外,还有许多其他类型的激活函数被用于神经网络。一些最常用的激活函数包括双曲正切函数、指数线性单元(ELU)函数和缩放指数线性单元(SELU)函数。

这些激活函数中的每一个都有自己的特性和使用情况,激活函数的选择取决于被解决的问题的具体需要。

激活函数是神经网络的一个关键组成部分,用于将非线性引入模型中。有许多不同类型的激活函数,每个都有自己的属性和使用情况。一些最常用的激活函数包括sigmoid函数、ReLU函数、leaky ReLU函数和softmax函数。

激活函数的选择取决于被解决的问题的具体需求,不同的激活函数可能更适合于不同类型的数据或任务。通过了解不同激活函数的特性,我们可以更好地设计神经网络,使其能够捕捉到现实世界数据中存在的复杂模式。


\subsection{损失函数优化}

损失函数优化是训练神经网络的一个重要组成部分。损失函数优化的目标是找到一组权重和偏置,使神经网络的预测输出与实际输出之间的差异最小。

在本节中,我们将讨论损失函数的概念和它们在神经网络中的作用。我们还将讨论最常用的神经网络优化算法,包括梯度下降及其变种。

损失函数。

损失函数是一个数学函数,用来衡量神经网络的预测输出和实际输出之间的差异。损失函数的目标是提供一个衡量神经网络表现如何的标准。

损失函数的选择取决于正在解决的具体问题。例如,对于回归问题,平均平方误差(MSE)是一个常用的损失函数,而对于分类问题,通常使用交叉熵损失函数。

平均平方误差(MSE)损失函数由以下公式给出。

L = (1/n) * sum_i(y_i - y_hat_i)^2

其中n是数据集中的样本数,y_i是第i个样本的实际输出,y_hat_i是第i个样本的预测输出,和是在数据集中的所有样本中取的。

交叉熵损失函数通常用于分类问题,由以下公式给出。

L = - (1/n) * sum_i(y_i * log(y_hat_i) + (1 - y_i) * log(1 - y_hat_i)

其中n是数据集中的样本数,y_i是第i个样本的实际输出(二元分类为0或1,多类分类为单次编码向量),y_hat_i是第i个样本的预测输出(0和1之间的概率),和是在数据集中的所有样本中取值。

优化算法。

优化算法的目标是找到使损失函数最小化的权重和偏置集合。神经网络最常用的优化算法是梯度下降法。

梯度下降。

梯度下降是一种迭代优化算法,它沿着损失函数的负梯度方向更新神经网络的权重和偏置。负梯度指向最陡峭的下降方向,这意味着沿着这个方向更新权重和偏置将导致损失函数值的减少。

梯度下降的更新规则由以下公式给出。

w_i = w_i - learning_rate * dL/dw_i

其中w_i是第i个权重,learning_rate是决定更新步长的超参数,dL/dw_i是损失函数相对于第i个权重的偏导。

随机梯度下降法(Stochastic Gradient Descent)。

随机梯度下降(SGD)是梯度下降的一个变种,它基于随机选择的小型训练数据子集来更新权重和偏差。这样做的好处是在计算上比梯度下降法更有效率,因为它只需要计算一小部分数据的梯度。

SGD的更新规则与梯度下降相似,但使用在数据子集上计算的梯度而不是整个数据集。

w_i = w_i - learning_rate * dL_i/dw_i

其中dL_i/dw_i是损失函数相对于当前数据子集的第i个权重的偏导。


另一种流行的优化算法是亚当算法,它是一种自适应学习率优化算法。它使用梯度的第一和第二时刻的组合来动态调整训练期间的学习率。使用Adam更新参数的公式是。

m_t = β1 * m_t-1 + (1 - β1) * g_t

v_t = β2 * v_t-1 + (1 - β2) * (g_t^2)

m_hat = m_t / (1 - β1^t)

v_hat = v_t / (1 - β2^t)

θ_t+1 = θ_t - α * m_hat / (√v_hat + ε)

其中m_t和v_t是梯度的第一和第二时刻,β1和β2是第一和第二时刻的衰减率,g_t是损失函数在时间t的梯度,m_hat和v_hat是第一和第二时刻的偏差修正估计值,ε是一个小常数,以防止被零除,θ_t+1是更新的参数向量。

总之,损失函数优化是深度学习的一个关键方面,因为它直接影响到神经网络的性能。通过使用各种优化算法,如梯度下降、SGD和Adam,我们可以训练我们的神经网络来最小化损失函数,并在各种任务中获得高精确度。


\section{深度强化学习}

\subsection{深度Q学习}

\subsection{近端策略优化}

\subsection{深度确定性策略梯度}
