\chapter{总结与展望}
\label{chp:version_license}

\section{工作总结}

人们对出行效率和出行体验的要求越来越高,交通问题也日益突出。传统的交通管理方法已经难以应对日益增长的交通需求和不断变化的交通状况。因此,研究如何更好地优化交通流动,提高个体出行效用,是当前交通领域的重要研究方向之一。传统的出行选择模型主要基于随机效用理论,对人们的出行行为进行描述和预测。然而,这些模型忽略了个体对交通环境的实时感知和对环境变化的适应能力,因此其预测准确性有限。而基于深度强化学习的出行选择模型则可以在动态交通环境中实时调整个体的决策策略,提高出行效用,并且具有更高的预测准确性和适应能力。本文提出了一种基于深度强化学习的新型出行模式与时间选择模型,旨在解决传统模型的局限性,并实现更高效、更智能的出行决策。该模型能够适应复杂的交通环境,并且可以处理许多具有出行决策请求的个体,具有较高的计算效率和优秀的性能表现。该研究成果不仅可以为交通管理提供新的思路和方法,还可以为人们提供更高效、更个性化的出行选择建议,提高人们的出行体验和生活质量。论文的主要研究如下:

(1)基于SUMO的城市多模式路网场景

本研究探讨了城市交通多模式仿真环境的建立,主要包括路网编辑与生成、出行模式设计和流量生成三个方面。通过对SUMO搭建仿真路网存在的缺陷和不足进行分析,确保了实验场景能够满足不同研究需求。在出行模式设计部分,考虑了不同交通需求和策略,包括私家车、地铁、公交、自行车等,以便更好地反映城市交通的多样性。在流量生成部分,根据实际数据生成合理的交通流量,提供了准确的交通分析。通过该多模式仿真环境的建立,为交通规划、智慧交通等相关领域的研究和实践提供了可靠的仿真平台。

(2)出行模式与时间选择问题特定的马尔可夫决策过程

马尔可夫决策模型可以充分考虑不同状态之间的转移概率。在出行模式和时间选择问题中,个体每天的状态可能受到多种因素的影响,例如天气、交通状况等,而这些因素的变化可能会影响个体决策。因此,马尔可夫决策模型可以对这些状态进行建模,并考虑它们之间的转移概率,以更好地理解和解决这些问题。其将整个决策过程形式化为一个数学问题,通过定义状态空间、动作空间和奖励函数,马尔可夫决策框架将出行模式和时间选择问题转化为一个数学问题,从而方便进一步的理论研究和算法优化。


(3)基于深度强化学习的出行模式与时间选择模型

为了在处理出行数据时能够更好地提取和泛化特征,以提高模型的准确性和稳定性。本研究使用了基于聚类的深度强化学习方法,该方法使用聚类算法将出行数据集分成不同的子集,然后对每个子集进行深度强化学习模型的训练。这样做的好处是可以在每个子集中学习更具体和相关的特征,从而提高模型的预测能力。此外,该方法还通过改进深度强化学习模型来进一步提高其准确性和稳定性。整个模型的目的是为了能够更好地解决出行模式与时间选择问题,提高个体的出行效用。

(4)针对改进的深度强化学习方法的训练和评估

深度强化学习方法在解决交通出行中的多目标决策问题方面具有很好的应用潜力和实际意义。但是,深度强化学习方法的训练和评估是一个复杂的过程,需要针对实际问题进行合适的实验场景设置、模型参数调优、训练过程监控和结果分析,才能得出具有参考价值的研究成果。训练和评估可以帮助研究者更好地了解模型的性能、泛化能力和灵敏性等方面的特征,以及与其他方法的比较优劣。本研究对实验结果进行了全面的模型评估,包括与传统DQN方法的对比、各测试个体的性能分析以及模型的泛化能力和灵敏性分析。最终,研究得出的结论是基于深度强化学习的模式与出发时间选择方法在交通出行领域中具有很好的应用价值和实际意义,可以更快地收敛到最优解,并且模型的泛化能力和灵敏性得到了验证。


\section{论文创新点}

本文的创新点如下:

(1)联合出行模式和出发时间选择问题建模为连续多天的马尔可夫决策过程,并用深度强化学习模型求解。

传统的出行模式选择和出发时间选择问题通常被视为静态决策问题,即每次出行都是一个单独的决策过程。但实际上,人们的出行决策往往会受到历史决策的影响,因此将这些决策过程建模为连续多天的马尔科夫决策过程模型是更为真实和合理的。深度强化学习是一种强化学习的方法,可以通过让模型自主学习和改进来解决复杂的决策问题。与传统的规则或手动设计模型相比,深度强化学习模型可以更好地适应实际情况和变化,从而提高模型的效果和性能。

(2)提出一种新的深度强化学习方法,利用聚类算法选取代表性个体进行高效训练,作为解决算法。

本研究提出的基于深度强化学习方法是为了解决多模式出行模式和出发时间选择问题而设计的,而传统的深度强化学习方法的训练过程通常需要大量的训练数据和计算资源。因此,本研究引入了一种新的深度强化学习方法,该方法使用聚类算法对大量个体进行聚类,并从每个聚类中选取代表性个体进行训练。这种方法可以大大减少训练数据和计算资源的需求,提高了训练效率和训练效果。这是本研究提出的一项创新点,也是本研究能够成功解决多模式出行模式和出发时间选择问题的关键因素之一。

(3)在真实城市交通网络上进行多模式微观仿真实验,以展示与验证所提出的方法的有效性。

在交通规划和智能交通领域,很多方法和算法都是基于理论或简化的仿真场景进行设计和验证的。然而,在真实世界的城市网络中,存在许多复杂的因素,如道路拓扑结构、交通信号控制、出行行为等等,这些因素对交通流和出行模式的产生和变化都具有重要的影响。因此,对于交通领域的研究来说,在真实世界网络上进行多模式微观仿真实验是非常重要的,可以更准确地反映出真实的交通情况,并验证所提出的方法的有效性和可行性。本文在真实世界网络上进行了多模式微观仿真实验,并与其他方法进行比较和敏感性分析,验证了所提出方法的优越性和鲁棒性。

\section{展望}

在这项研究中提出了一种深度强化学习方法,为在高峰时段(早上7点至9点)出行的个体提供更好的出行选择。该方法旨在在考虑不同出行方式的同时,最大程度地减少用户的出行时间和成本。研究结果表明,该方法在提供准确、高效的出行建议方面是有效的。这种方法的一个关键优势是其灵活性,因为奖励函数可以根据不同用户或情况的特定需求和要求进行调整。但由于学术水平和时间精力有限,仍存在一些问题值得进一步思考与完善,主要包括:

(1)当前模型是为单个旅行者出行选择推荐而设计的,并没有考虑多个旅行者对交通系统的潜在影响。当前模型的设计是出于简化模型和降低计算复杂度的考虑,因此只考虑了单个旅行者的出行选择。但是,在实际交通系统中,多个旅行者的出行选择会相互影响,可能会导致交通系统的拥堵或效率低下。因此,未来研究可以考虑将模型扩展到多智能体框架,以考虑多个旅行者对交通系统的潜在影响。这种扩展可以提高模型的现实性和适用性,进一步优化交通系统的性能,并对城市交通规划和管理提供更好的指导和支持。

(2)该研究使用了一天的出行需求数据作为输入,以提供第二天的出行建议。这意味着该模型无法提供实时的推荐决策,因为需要等待一天的数据输入。未来的研究可以探索如何整合实时数据以建立实时推荐系统,以便在实时交通拥堵或其他情况下,及时为旅客提供最优的出行建议。这可以通过结合实时交通数据、用户实时出行意向以及其他相关信息来实现。

(3)未来研究可以探索如何将个体的行为和社会人口特征纳入到建模框架中,并考虑将换乘作为一种附加的交通方式纳入到决策过程中。同时,需要研究如何在出行选择方面考虑个体之间的动态互动和合作,以提高整个系统的效率和利用交通网络资源。